% \documentclass{article}
% \usepackage{graphicx} % Required for inserting images
% \usepackage[english,russian]{babel}
% \usepackage[T1,T2A]{fontenc}
% \usepackage[utf8]{inputenc}
% \usepackage{amsmath} % для использования символа pi и прочих математических символов
\documentclass{article}
\usepackage[utf8]{inputenc}
\usepackage{amsmath}
\usepackage{amsfonts}
\usepackage[T2A]{fontenc}
\usepackage[russian]{babel}
\usepackage[left=2cm,right=2cm,
    top=2cm,bottom=2cm]{geometry}
\usepackage{graphicx}
\usepackage{MnSymbol}%
\usepackage{wasysym}%
\pagenumbering{gobble}
\usepackage{listings}
\title{Lab}
\author{Даниил К}
\date{March 2023}
\selectlanguage{russian}
\begin{document}
\section*{Аналитическая часть}
\subsection*{Построить верхнюю и нижнюю суммы Дарбу для равномерного разбиения (на n частей)}

Для функции $f(x)=\sin^2(2x)$ и равномерного разбиения интервала $[a,b]$ на $n$ частей, верхняя и нижняя суммы Дарбу могут быть записаны в следующем виде:
Нижняя сумма Дарбу:
$$s_\tau(f) = \frac{b-a}{n}\sum_{i=1}^n m_i \Delta x_i = \frac{b-a}{n}\sum_{i=1}^{n}\left(\inf_{x\in[x_{i-1},x_i]} f(x)\right)
$$$$= \frac{b-a}{n}\sum_{i=1}^{n}\left(\sin^2\left(2\frac{a+i-1}{n}\pi\right)\right)$$
Верхняя сумма Дарбу:
$$S_\tau(f) = \frac{b-a}{n}\sum_{i=1}^n M_i \Delta x_i =
\frac{b-a}{n}\sum_{i=1}^{n}\left(\sup_{x\in[x_{i-1},x_i]} f(x)\right)
$$$$= \frac{b-a}{n}\sum_{i=1}^{n}\left(\sin^2\left(2\frac{a+i}{n}\pi\right)\right)$$

Здесь $a$ - левый конец отрезка, а $b$ - правый конец отрезка $[a;b]$, 
$x_{i-1}=a+(i-1)\frac{b-a}{n}$ и $x_i=a+i\frac{b-a}{n}$ обозначают границы $i$-го подотрезка, а $\inf_{x\in[x_{i-1},x_i]} f(x)$ и $\sup_{x\in[x_{i-1},x_i]} f(x)$ являются инфинум и супремум значением функции $f(x)$ на $i$-ом подотрезке соответственно.\\
Для подсчёта сумм воспользуемся известной формулой суммы квадратов синусов для конечного набора углов с помощью формулы:
$$\sum_{i=1}^n \sin^2(\phi_i) = \frac{n}{2} - \frac{1}{2}\sum_{i=1}^n \cos(2\phi_i)\text{, где } \phi_i \text{ это угол из набора}$$

Перепишем формулы сумм, учитывая данный в условии отрезок $[0; 2\pi]$:\\

$$s_\tau(f) = \frac{2\pi-0}{n}\sum_{i=1}^{n}\left(\sin^2\left(2\frac{i-1}{n}\pi\right)\right)
= \frac{2\pi}{n} \left( \frac{n}{2} - \frac{1}{2}\sum_{i=1}^n \cos(4 \frac{(i-1)\pi}{n})\right)  $$
$$ = \frac{2\pi}{n} \left( \frac{n}{2} - 0\right) = \frac{2\pi}{n}\cdot\frac{n}{2} = \boxed{\pi}$$

$$S_\tau(f) = 
\frac{2\pi-0}{n}\sum_{i=1}^{n}\left(\sin^2\left(2\frac{i}{n}\pi\right)\right)  = 
\frac{2\pi}{n} \left( \frac{n}{2} - \frac{1}{2}\sum_{i=1}^n \cos(4 \frac{i\pi}{n})\right)   =  $$ 
$$ = \frac{2\pi}{n} \left( \frac{n}{2} - 0\right) = \frac{2\pi}{n}\cdot\frac{n}{2} = \boxed{\pi}$$

\subsection*{Проверить критерий Римана интегрируемости функции, сделать вывод. Как ещё можно доказать интегрируемость данной функции?}
\text{Критерий интегрируемости функции Римана:}
$$\forall \epsilon > 0 \exists \tau : S_\tau-s_\tau < \epsilon$$
Проверим для данной в условии функции, учитывая найденные ранее $s_\tau$ и $S_\tau$:
$$ S_\tau - s_\tau = \pi - \pi = 0 $$
Тривиально, что это будет меньше, чем $\forall \epsilon$ > 0. А значит, получили, что критерий Римана выполняется для $f(x)=\sin^2(2x)$ на отрезке $[0; 2\pi]$.\\
Ещё способы доказать интегрируемость $f(x)=\sin^2(2x)$ на отрезке $[0; 2\pi]$:\\
Теорема об интегрируемости непрерывной функции:\\
$$f(x)=\sin^2(2x) \in C[0; 2\pi] \Rightarrow f(x)=\sin^2(2x) \in R[0; 2\pi]$$\\
Теорема о конечном числе точек разрыва:\\
Для доказательства интегрируемости функции $f(x)=\sin^2(2x)$ на отрезке $[0,2\pi]$ достаточно показать, что она является ограниченной на данном отрезке и имеет только конечное число точек разрыва первого рода.

Ограниченность функции $f(x)$ на отрезке $[0,2\pi]$ следует из того, что $0 \leq \sin^2(2x) \leq 1$ для любого $x \in [0,2\pi]$. Таким образом, функция $f(x)$ ограничена на отрезке $[0,2\pi]$ сверху и снизу числами 1 и 0.

Далее, функция $f(x)$ имеет только конечное число точек разрыва первого рода на отрезке $[0,2\pi]$, поскольку $\sin(2x)$ имеет только конечное число точек разрыва первого рода на этом отрезке, а квадрат функции $\sin(2x)$ также имеет только конечное число таких точек.

Таким образом, функция $f(x)$ удовлетворяет условиям теоремы, и следовательно, является интегрируемой на отрезке $[0,2\pi]$.

\subsection*{Найти пределы сумм Дарбу, сделать вывод о значении интеграла.}
$$\lim\limits_{\lambda (\tau) \to 0} S_\tau =\lim\limits_{\lambda (\tau) \to 0} \frac{2\pi}{n}\frac{n}{2}  =\lim\limits_{\lambda (\tau) \to 0} \pi = \pi$$

$$\lim\limits_{\lambda (\tau) \to 0} s_\tau =\lim\limits_{\lambda (\tau) \to 0} \frac{2\pi}{n}\frac{n}{2} =\lim\limits_{\lambda (\tau) \to 0} \pi = \pi$$
Логичный вывод, что $I = \pi$
\subsection*{Проверить результат с помощью формулы Ньютона — Лейбница.}
$$\int\limits_0^{2\pi} \sin^2(2x) dx = \frac{1}{2}\int\limits_0^{2\pi} 1 - \cos(4x) dx = \frac{1}{2} \left(x - \frac{1}{4}\sin(4x)\right)\bigg|_0^{2\pi} = \boxed{\pi}$$
Получили, что пределы сумм Дарбу совпадают с определенным интегралом, посчитанным по формуле Ньютона-Лейбница.
\section*{Численный метод}
\begin{table}[h]
    \centering
    \caption{Результаты}
    \begin{tabular}{|с|c|c|}
    \hline
        Заголовок 1 & Заголовок 2 & Заголовок 3 \\ \hline
        Строка 1, ячейка 1 & Строка 1, ячейка 2 & Строка 1, ячейка 3 \\ \hline
        Строка 2, ячейка 1 & Строка 2, ячейка 2 & Строка 2, ячейка 3 \\ \hline
        Строка 3, ячейка 1 & Строка 3, ячейка 2 & Строка 3, ячейка 3 \\ \hline
    \end{tabular}
    \label{tab:example}
\end{table}

left: 3.141593
right: 3.141593
midpoint: 3.141593
random: 3.252122
trapezoidal: 3.141593

\begin{lstlisting}[language=Python]
# Здесь ваш код на языке        Python(3.10.1)

\end{lstlisting}

% \lstinputlisting[language=Python]{code.py}
\end{document}
